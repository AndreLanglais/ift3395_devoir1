\documentclass{article}

\usepackage[utf8]{inputenc}
\usepackage{indentfirst}
\usepackage{amsmath}

\title{Devoir \#1 - IFT3395/6390}
\author{Julien Allard et André Langlais}
\date{12 Octobre 2017}

\begin{document}
	
	\maketitle
	\newpage
	\section{Petit exercice de probabilités}
	\indent Soit les variables aléatoires suivantes: \\	
	 		X: Femme atteinte du cancer \\ 		 			
	 		Y: Test de détection est positif \\
	 		
	 Formule de Bayes:
	 	
	 \begin{equation}
		P(X|Y)= \frac{P(X)*P(Y|X)}{P(Y)}
	 \end{equation}
	 	
	 	
	 Les données de la question nous permettent de trouver les probabilités suivantes:
	 \begin{equation}
	 	\begin{split}
	 	& P(X) = 0.01 \\
	 	& P(Y|X)  = 0.8 \\
	 	\end{split}
	 \end{equation}
	 
	 Il nous manque à trouver {P(Y)} qui peut être calculer par la formule suivante:
	 \begin{equation}
	 P(Y) = P(Y|X) + P(Y|X')
	 \end{equation}
	 
	 où X' répresente l'évènement complémentaire de X
	 
	 \begin{equation}
	 P(Y)= 0.01*0.80 + 0.99*0.096 = 0.10304
	 \end{equation}
	 
	 Nous avons maintenant toutes les informations requises pour appliquer le théorème de Bayes
	 
	 	 \begin{equation}
	 	 P(X|Y)= \frac{0.01*0.80}{0.10304} \approx 0.0774
	 	 \end{equation}
	 
	 La bonne réponse est donc 6. soit moins que 10\%. 	 
	 Il peut paraître surprenant que le pourcentage soit si bas, mais on peut vérifier visuellement pourquoi on obtient un tel pourcentage
	
	
	\section{Estimation de densité : paramétrique Gaussienne,
		v.s. fenêtres de Parzen}
	\subsection{Gaussienne isotropique}
	
	\subsubsection*{(a)}
	Les paramètres sont la moyenne $\mu$ de dimension d et la variance $\sigma$ de dimension 1 \\
	
	
	\subsubsection*{(b)}
	\begin{equation}
	\begin{split}
		n &= |D| \\
		\mu_{MaxVraiss} &= \sum_{i}^{n} \frac{x_{i}}{nd} \\
		\sigma_{MaxVraiss}^{2} &= \sum_{i}^{n} \frac{(x_{i}- \mu)^{T}(x_{i} - \mu)}{nd} \\
	\end{split}
	\end{equation}
	
	\subsubsection*{(c)}
	$\mu$ se calcule en $O(nd)$  et $\sigma^{2} $ se calcule en $O(nd)$ aussi\\
	\subsubsection*{(d)}
	\begin{equation}
	\frac{1}{2\pi^{\frac{d}{2}}\sigma^d}\exp^{\frac{-1}{2}\frac{x-\mu}{dénominateur}
	\end{equation}
	
	\subsubsection*{(e)}
	
	
	\subsection{Fenêtres de Parzen avec noyau Gaussien isotropique}
	
	\subsubsection*{(a)}
	
	\subsubsection*{(b)}
	
	\subsubsection*{(c)}
	
	
	\subsection{Capacité}
	
	\subsubsection*{(a)}
	
	\subsubsection*{(b)}
	
	\subsubsection*{(c)}
	
	
	\subsection{Gaussienne diagonale}
	
	\subsubsection*{(a)}
	
	\subsubsection*{(b)}
	
	\subsubsection*{(c)}
	
	\subsubsection*{(d)}
	
	
	\subsection{Problème de classification}
		
	\subsubsection*{(a)}
		
	\subsubsection*{(b)}
	
\end{document}\
